%-------------------------
% Resume in Latex
% Author : Gennadii Chursov
% Based off of: https://github.com/sb2nov/resume and https://www.overleaf.com/latex/templates/jakes-resume/syzfjbzwjncs
% License : MIT
%------------------------

\documentclass[letterpaper,11pt]{article}

\usepackage{latexsym}
\usepackage[empty]{fullpage}
\usepackage{titlesec}
\usepackage{marvosym}
\usepackage[usenames,dvipsnames]{color}
\usepackage{verbatim}
\usepackage{enumitem}
\usepackage[hidelinks]{hyperref}
\usepackage{fancyhdr}
\usepackage[english]{babel}
\usepackage{tabularx}
\usepackage{ragged2e}
\input{glyphtounicode}


%----------FONT OPTIONS----------
% sans-serif
% \usepackage[sfdefault]{FiraSans}
% \usepackage[sfdefault]{roboto}
% \usepackage[sfdefault]{noto-sans}
% \usepackage[default]{sourcesanspro}

% serif
% \usepackage{CormorantGaramond}
% \usepackage{charter}


\pagestyle{fancy}
\fancyhf{} % clear all header and footer fields
\fancyfoot{}
\renewcommand{\headrulewidth}{0pt}
\renewcommand{\footrulewidth}{0pt}

% Adjust margins
\addtolength{\oddsidemargin}{-0.5in}
\addtolength{\evensidemargin}{-0.5in}
\addtolength{\textwidth}{1in}
\addtolength{\topmargin}{-.5in}
\addtolength{\textheight}{1.0in}

\urlstyle{same}
\hypersetup {
  colorlinks=false,% hyperlinks will be black
  pdfborderstyle={/S/U/W 1}% border style will be underline of width 1pt
}

\raggedbottom
\raggedright
\setlength{\tabcolsep}{0in}

% Sections formatting
\titleformat{\section}{
  \vspace{-4pt}\scshape\raggedright\large
}{}{0em}{}[\color{black}\titlerule \vspace{-5pt}]

% Ensure that generate pdf is machine readable/ATS parsable
\pdfgentounicode=1

%-------------------------
% Custom commands
\newcommand{\resumeItem}[1]{
  \item\small{
    {#1 \vspace{-2pt}}
  }
}

\newcommand{\resumeSubheading}[4]{
  \vspace{-2pt}\item
    \begin{tabular*}{0.97\textwidth}[t]{l@{\extracolsep{\fill}}r}
      \textbf{#1} & #2 \\
      \textit{\small#3} & \textit{\small #4} \\
    \end{tabular*}\vspace{-7pt}
}

\newcommand{\resumeSubSubheading}[2]{
  \item
  \begin{tabular*}{0.97\textwidth}{l@{\extracolsep{\fill}}r}
    \textit{\small#1} & \textit{\small #2} \\
  \end{tabular*}\vspace{-7pt}
}

\newcommand{\resumeProjectHeading}[3]{
  \item
  \begin{tabular*}{0.97\textwidth}{l@{\extracolsep{\fill}}r}
    \small\textbf{#1} & #2 \\
    \textit{\small#3} \\
  \end{tabular*}\vspace{-7pt}
}

\newcommand{\resumeSubItem}[1]{\resumeItem{#1}\vspace{-4pt}}

\renewcommand\labelitemii{$\vcenter{\hbox{\tiny$\bullet$}}$}

\newcommand{\resumeSubHeadingListStart}{\begin{itemize}[leftmargin=0.15in, label={}]}
\newcommand{\resumeSubHeadingListEnd}{\end{itemize}}
\newcommand{\resumeItemListStart}{\begin{itemize}}
\newcommand{\resumeItemListEnd}{\end{itemize}\vspace{-5pt}}

%-------------------------------------------
%%%%%%  RESUME STARTS HERE  %%%%%%%%%%%%%%%%%%%%%%%%%%%%

\begin{document}

\begin{center}
  \textbf{\Huge \scshape Fuad Ismail} \\ \vspace{1pt}
  \small \textit{} \\
  +62-838-9937-3595 $|$ \href{mailto:fuad1502@gmail.com}{fuad1502@gmail.com} $|$ 
  \href{https://github.com/fuad1502}{github.com/fuad1502}
\end{center}

%-----------SUMMARY-----------
\section*{Summary}
\justifying
I am an ex-R\&D Hardware Engineer from Keysight Technologies Malaysia with over
4 years of experience in the hardware industry (RF, electronics, embedded, and
communication systems). Despite that, I am also acquinted with software
development, as my work frequently require me to develop softwares, such as
embedded software, mathematical simulation, and computational libraries.

I took a career break since August 2023 due to personal reasons. I am currently
self studying Computer Science to break into the industry. See
\href{https://github.com/fuad1502/self-study-cs}{github.com/fuad1502/self-study-cs}
to know what I've achieved throughout my 7-months career break.

I am mainly interested in Systems Software (OS, Database, Distributed and
Embedded Systems) and Computing Hardware. My goal is to consistently contribute
to the OSS ecosystem by contributing to public repositories and building my own
OSS projects. Currently, I have contributed to
\href{https://github.com/verilator/verilator}{Verilator}, an open-source
SystemVerilog simulator and lint system. (Pull request
\href{https://github.com/verilator/verilator/pull/5006}{\#5006},
\href{https://github.com/verilator/verilator/pull/4966}{\#4966})

I am a passionate learner committed to continuous learning and have a track
record of showing initiative at the workplace and deeply care about my work.

%-----------EDUCATION-----------
\section{Education}
\resumeSubHeadingListStart
\resumeSubheading
  {Bandung Institute of Technology}{Bandung, Indonesia}
  {Electrical Engineering (BS), ABET Accredited, GPA
  3.81/4.0}{2014 - 2018}
\resumeSubHeadingListEnd

%-----------PROGRAMMING SKILLS-----------
\section{Technical Skills}
\begin{itemize}[leftmargin=0.15in, label={}]
\small{\item{
  \textbf{Programming Languages}{: SystemVerilog/Verilog, C, C++, MATLAB, Rust,
    Go, Java, JavaScript, Python, SQL} \\
  \textbf{Softwares}{: Vivado, Quartus/ModelSim, Verilator, KiCad, Mentor
    Graphics Xpedition} \\
  \textbf{Hardwares}{: Xilinx Zynq APSoC, STMicroelectronics STM32} \\
  \textbf{Developer Tools}{: Git, Docker/Podman, CMake, Make}
}}
\end{itemize}

%-----------PROJECTS-----------
\section{Projects}
\resumeSubHeadingListStart
\resumeProjectHeading
  {RVSV}{November 2023 -- Present}
  {\href{https://github.com/fuad1502/rvsv}{github.com/fuad1502/rvsv}}
\begin{justify}
  RVSV is a SystemVerilog implementation of a 5-stage pipelined RISC-V CPU.
  Verification code is written in C++ using Verilator and Rubbler.
\end{justify}

\resumeProjectHeading
  {Rubbler}{November 2023 -- Present}
  {\href{https://github.com/fuad1502/rubbler}{github.com/fuad1502/rubbler}}
\begin{justify}
  Rubbler is a RISC-V assembler written in Rust. This library was written with
  the main purpose of embedding a simple RISC-V assembler inside of a RISC-V
  CPU test bench code written with Verilator.
\end{justify}

\resumeProjectHeading
  {NAALG: Network Analyzer Algorithms C++ Library}{}
  {Associated with Keysight Technologies}
\begin{justify}
  I initiated the development of NAALG due two things. First, I was tasked with
  developing a novel Time Domain Reflectometry (TDR) calculation from Vector
  Network Analyzer (VNA) measurements. Second, on a different project, I had
  just implemented a new calibration algorithm for an embedded VNA. Since both
  of this project involves calculation on VNA measurements, I realize that it
  would be nice to have a library that provides all of these VNA algorithms in
  one place. NAALG is similar to
  \href{https://scikit-rf.readthedocs.io}{scikit-rf}, but uses high performant
  implementation in C++ to enable usage in embedded real-time measurement
  applications.
\end{justify}

\resumeProjectHeading
  {Custom Telecommunication Device (GMSK + TDM Transceiver)}{}
  {Associated with Hariff Daya Tunggal Engineering}
\begin{justify}
  My team was tasked with developing a fully custom telecommunication device.
  My role was in evaluating system level design tradeoffs using MATLAB and
  developing the physical (PHY) and medium access control (MAC) layer on an
  \textit{All Programmable SoC (APSoC)}. The PHY layer consists of, but not
  limited to, baseband modulator, channel impulse response estimator, and
  Viterbi equalizer. The PHY layer is written in Verilog, while the MAC layer
  is written in C. Communication between the two layers uses Direct Memory
  Access (DMA).
\end{justify}

\resumeProjectHeading
  {Automatic Identification System Search and Rescue Transponder}{}
  {Associated with Labs247}
\begin{justify}
  Automatic Identification System (AIS) Search and Rescue Transponder (SART) is
  a radio device used to locate distressed vessels. My role was in developing
  all of the electronic and embedded software aspect of the device, and
  production. We use \href{https://freertos.org}{FreeRTOS} Real-Time Operating
  System to enable low power usage and multiple tasks management. Up till now
  the device is still in production.
\end{justify}

\resumeProjectHeading
  {Physical, Data Link, and Network Layer Implementation for Visible Light
  Communication}{}
  {Associated with Institut Teknologi Bandung}
\begin{justify}
  Visible Light Communication (VLC), or sometimes referred as LiFi, is a
  communication system that aims to use regular LEDs used in for lighting as an
  internet access point. An IEEE standard has been released to accommodate the
  standardization of this emerging technology, namely IEEE 802.15.7. In this
  project we have made a fully functional VLC access point for a single user
  that has a data rate of around 500 kbps, enabling streaming YouTube videos in
  144p. My workload was to create the baseband signal processing on an All
  Programmable SoC, which meant the design utilizes both an FPGA and a
  processor. The network and data link layer are implemented with a program
  written in C. The physical layer was designed based on the IEEE 802.15.7
  standard, which includes VPPM encoding, CRC append and check, and
  Reed-Solomon FEC. The physical layer ran on the FPGA substrate and was
  written in Verilog. I also integrated a Xillybus IP core for the
  interconnection between the FPGA and the processor.
\end{justify}
\resumeSubHeadingListEnd

%-----------EXPERIENCE-----------
\section{Experience}
\resumeSubHeadingListStart
\resumeSubheading
  {R\&D Hardware Engineer}{January 2022 -- August 2023}
  {Keysight Technologies}{Penang, Malaysia}
\resumeItemListStart
\resumeItem{Initiated the development of NAALG (\textit{see above}), a network
analyzer algorithms C++ library.}
\resumeItem{Significantly improve the measurement accuracy of an embedded VNA
product by introducing and implementing a better calibration algorithm.}
\resumeItem{Successfully extend the measurement bandwidth of an embedded VNA
product by suggesting improvements to the RF layout.}
\resumeItem{Fixed an RF technical issue in an embedded VNA product and wrote a
technical paper on it.}
\resumeItem{Demonstrated iniatiative by voluntarily improving an internal
software tool used by our team.}
\resumeItem{Demonstrated initiative by introducing a better software
development workflow.}
\resumeItem{Collaborate in a team from various time zones (USA, Europe, and
Malaysia).}
\resumeItemListEnd

\resumeSubheading
  {Software Engineer}{February 2021 - October 2021}
  {Lumina Industries}{Jakarta, Indonesia}
\resumeItemListStart
\resumeItem{Wrote the Windows middleware for a Virtual Camera software using
Win32 API.}
\resumeItem{Developed image processing features using C++ with OpenCV.}
\resumeItem{Worked remotely with a medium-sized team from various nationalities
(USA, Taiwan, and Indonesia).}
\resumeItemListEnd

\resumeSubheading
  {R\&D Radio Frequency Engineer}{August 2020 -- February 2021}
  {Hariff Daya Tunggal Engineering}{Bandung, Indonesia}
\resumeItemListStart
\resumeItem{Wrote MATLAB simulations to evaluate system-level design tradeoffs
for the development of a fully custom telecommunication device (\textit{see
above}).}
\resumeItem{Implemented the PHY and MAC layer of a fully custom
telecommunication device on an APSoC using Verilog and C.}
\resumeItem{Designed a schematic for a fully custom telecommunication device.}
\resumeItemListEnd

\resumeSubheading
  {R\&D Electronic Design Engineer}{August 2018 -- August 2020}
  {Labs247}{Jakarta, Indonesia}
\resumeItemListStart
\resumeItem{Lead the development of telecommunication product (Automatic
Identification System Search and Rescue Transponder, \textit{see above}) which
is still in production up till now.}
\resumeItem{Involved in the development of medical product (CPAP BiPAP
machine).}
\resumeItem{Demonstrated initiative by suggesting cost-saving solutions that
avoids vendor lock in and introduced an efficient production workflow.}
\resumeItemListEnd
\resumeSubHeadingListEnd

%-------------------------------------------
\end{document}
