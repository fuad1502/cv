%-------------------------
% Resume in Latex
% Author : Fuad Ismail
% Based off of: https://github.com/sb2nov/resume and https://www.overleaf.com/latex/templates/jakes-resume/syzfjbzwjncs
% License : MIT
%-------------------------

\documentclass[letterpaper,11pt]{article}

\usepackage{latexsym}
\usepackage[empty]{fullpage}
\usepackage{titlesec}
\usepackage{marvosym}
\usepackage[usenames,dvipsnames]{color}
\usepackage{verbatim}
\usepackage{enumitem}
\usepackage[hidelinks]{hyperref}
\usepackage{fancyhdr}
\usepackage[english]{babel}
\usepackage{tabularx}
\usepackage{ragged2e}
\usepackage{tipa}
\usepackage{relsize}
\input{glyphtounicode}

%-------------------------
% Font options
%-------------------------
% sans-serif
% \usepackage[sfdefault]{FiraSans}
% \usepackage[sfdefault]{roboto}
% \usepackage[sfdefault]{noto-sans}
% \usepackage[default]{sourcesanspro}

% serif
% \usepackage{CormorantGaramond}
% \usepackage{charter}

%-------------------------
% Page style
%-------------------------
\pagestyle{fancy}
\fancyhf{} % clear all header and footer fields
\fancyfoot{}
\renewcommand{\headrulewidth}{0pt}
\renewcommand{\footrulewidth}{0pt}

% Adjust margins
\addtolength{\oddsidemargin}{-0.5in}
\addtolength{\evensidemargin}{-0.5in}
\addtolength{\textwidth}{1in}
\addtolength{\topmargin}{-.5in}
\addtolength{\textheight}{1.0in}

\urlstyle{same}
\hypersetup {
  colorlinks=false,% hyperlinks will be black
  pdfborderstyle={/S/U/W 1}% border style will be underline of width 1pt
}

\raggedbottom
\raggedright
\setlength{\tabcolsep}{0in}

% Sections formatting
\titleformat{\section}{
  \vspace{-4pt}\scshape\raggedright\large
}{}{0em}{}[\color{black}\titlerule \vspace{-5pt}]

% Ensure that generate pdf is machine readable/ATS parsable
\pdfgentounicode=1

%-------------------------
% Custom Commands
%-------------------------
\newcommand{\resumeItem}[1]{
  \item\small{
    {#1 \vspace{-2pt}}
  }
}

\newcommand{\resumeSubheading}[4]{
  \vspace{-2pt}\item
    \begin{tabular*}{0.97\textwidth}[t]{l@{\extracolsep{\fill}}r}
      \textbf{#1} & #2 \\
      \textit{\small#3} & \textit{\small #4} \\
    \end{tabular*}\vspace{-7pt}
}

\newcommand{\resumeSubSubheading}[2]{
  \item
  \begin{tabular*}{0.97\textwidth}{l@{\extracolsep{\fill}}r}
    \textit{\small#1} & \textit{\small #2} \\
  \end{tabular*}\vspace{-7pt}
}

\newcommand{\resumeProjectHeading}[3]{
  \item
  \begin{tabular*}{0.97\textwidth}{l@{\extracolsep{\fill}}r}
    \small\textbf{#1} & #2 \\
    \textit{\small#3} \\
  \end{tabular*}\vspace{-7pt}
}

\newcommand{\resumeSubItem}[1]{\resumeItem{#1}\vspace{-4pt}}

\renewcommand\labelitemii{$\vcenter{\hbox{\tiny$\bullet$}}$}

\newcommand{\resumeSubHeadingListStart}{\begin{itemize}[leftmargin=0.15in, label={}]}
\newcommand{\resumeSubHeadingListEnd}{\end{itemize}}
\newcommand{\resumeItemListStart}{\begin{itemize}}
\newcommand{\resumeItemListEnd}{\end{itemize}\vspace{-5pt}}

% GitHub URL shorthands
\newcommand{\githubPRs}[2]{\href{https://github.com/#1/pulls?q=is\%3Apr+author\%3Afuad1502}{#2}}
\newcommand{\prLink}[2]{\href{https://github.com/#1/pull/#2}{\##2}}

% Other shorthands
\newcommand\CC{C\texttt{++}}
\newcommand\CS{C\texttt{\#}}

%-------------------------
% Document
%-------------------------
\begin{document}

%=========================
% Title
%=========================
\begin{center}
	\textbf{\Huge \scshape Fuad Ismail} \\
	\vspace{1pt} \small \textit{} \\
	+62-838-9937-3595 $|$ \href{mailto:fuad1502@gmail.com}{fuad1502@gmail.com} $|$ \href{https://github.com/fuad1502}{github.com/fuad1502}
\end{center}

%=========================
% Summary
%=========================
\section*{Summary}
\justifying

I have 2 years of experience as a software engineer and 4 years of R\&D
experience in the hardware industry. Currently, I am a software engineer at
Samsung, maintaining \& developing systems software running on Tizen OS and
independently working on optimizing the .NET runtime Just-In-Time compiler for
RISC-V architecture.

I have contributed to various \CC\ \& Rust open source projects:
\githubPRs{dotnet/runtime}{.NET runtime},
\githubPRs{verilator/verilator}{Verilator},
\githubPRs{uutils/coreutils}{uutils/coreutils}, and
\githubPRs{uutils/util-linux}{uutils/util-linux}. I am also currently working
on my own open source project:
\href{https://github.com/fuad1502/oombak}{Oombak}, an interactive SystemVerilog
simulator UI that runs on the terminal.

I am a self-motivated person, frequently showed initiative at the workpace. I
am also a passionate learner, currently interested in learning more about
compilers, operating systems, computer graphics, and embedded systems.

%=========================
% Education
%=========================
\section{Education}
\resumeSubHeadingListStart
\resumeSubheading
{Bandung Institute of Technology}{Bandung, Indonesia}
{Electrical Engineering (BSc), ABET Accredited, GPA 3.81/4.0}{2014 - 2018}
\resumeSubHeadingListEnd

%=========================
% Open Source Contributions
%=========================
\section{Open Source Contributions}
\begin{itemize}[leftmargin=0.15in, label={}]
	\small{\item{
	      \textbf{.NET runtime}{
		      : \prLink{dotnet/runtime}{112978},
		      \prLink{dotnet/runtime}{113250},
		      \prLink{dotnet/runtime}{113676},
		      \prLink{dotnet/runtime}{113999},
		      \prLink{dotnet/runtime}{114470},
		      \prLink{dotnet/runtime}{114488},
		      \prLink{dotnet/runtime}{114728},
		      \prLink{dotnet/runtime}{117408}
	      } \\
	      \textbf{Verilator}{
		      : \prLink{verilator/verilator}{4966},
		      \prLink{verilator/verilator}{5006}
	      } \\
	      \textbf{uutils/coreutils}{
		      : \prLink{uutils/coreutils}{6951}
	      } \\
	      \textbf{uutils/util-linux}{
		      : \prLink{uutils/util-linux}{152},
		      \prLink{uutils/util-linux}{160},
		      \prLink{uutils/util-linux}{167}
	      }
	      }}
\end{itemize}

%=========================
% Experiences in S/W Industry
%=========================
\section{Experiences in S/W Industry}
\resumeSubHeadingListStart

\resumeSubheading
{\href{https://www.samsung.com/id/srin/}{Samsung}}{June 2024 - Present}
{Software Engineer (TV S/W Platform)}{Jakarta, Indonesia}
\resumeItemListStart
\resumeItem{
	Demonstrated intiative by independently introducing multiple
	optimizations to upstream .NET runtime RISC-V Just-In-Time compiler
	which improved multiple micro-benchmarks performance scores on
	Samsung's RISC-V TV chipset by up to 70\%.
}
\resumeItem{
	Resolve various issues and implement new features for Tizen OS's App
	Framework \& .NET Framework systems software (e.g. installer \&
	application launcher) written in \CC.
}
\resumeItem{Involved in the development of a new Tizen OS daemon written in
	\CC.}
\resumeItem{Lead the development of an in-house Tizen .NET application written
	in \CS.}
\resumeItem{Initiated the porting of a systems component to Rust.}
\resumeItem{Received an employee of the quarter award.}
\resumeItemListEnd

\resumeSubheading
{\href{https://getlumina.com/}{Lumina Industries}}{February 2021 - October 2021}
{Software Engineer}{Jakarta, Indonesia}
\resumeItemListStart
\resumeItem{Wrote the Windows middleware for a Virtual Camera software using
	Win32 API.}
\resumeItem{Developed image processing features using \CC\ with OpenCV.}
\resumeItem{Worked remotely in a medium-sized team from various nationalities
	(USA, Taiwan, and Indonesia).}
\resumeItemListEnd

\resumeSubHeadingListEnd

%=========================
% Experiences in H/W Industry
%=========================
\section{Experiences in H/W Industry}
\resumeSubHeadingListStart

\resumeSubheading
{\href{https://www.keysight.com/}{Keysight Technologies}}{January 2022 -- August 2023}
{R\&D Hardware Engineer}{Penang, Malaysia}
\resumeItemListStart
\resumeItem{Initiated the development of NAALG, a network analyzer algorithms
	\CC\ library.}
\resumeItem{Significantly improve the measurement accuracy of an embedded VNA
	product by introducing and implementing a better calibration algorithm.}
\resumeItem{Successfully extend the measurement bandwidth of an embedded VNA
	product.}
\resumeItem{Fixed a technical issue in an embedded VNA product and wrote a
	technical paper on it.}
\resumeItem{Collaborate in a team from various time zones (USA, Europe, and
	Malaysia).}
\resumeItemListEnd

\resumeSubheading
{\href{https://hariff.co.id/}{Hariff Daya Tunggal Engineering}}{August 2020 -- February 2021}
{R\&D Radio Frequency Engineer}{Bandung, Indonesia}
\resumeItemListStart
\resumeItem{Wrote MATLAB simulations to evaluate system-level design tradeoffs
	for the development of a fully custom telecommunication device.}
\resumeItem{Implemented the PHY and MAC layer of a fully custom telecommunication
	device on an APSoC using Verilog and C.}
\resumeItem{Designed a schematic for a fully custom telecommunication device.}
\resumeItemListEnd

\resumeSubheading
{Labs247 (\href{https://solusi247.com}{SOLUSI247})}{August 2018 -- August 2020}
{R\&D Electronic Design Engineer}{Jakarta, Indonesia}
\resumeItemListStart
\resumeItem{Lead the development of telecommunication product (AIS SART) which
	is still in production up till now.}
\resumeItem{Involved in the development of medical product (CPAP BiPAP
	machine).}
\resumeItem{Demonstrated initiative by suggesting cost-saving solutions that
	avoids vendor lock in and introduced an efficient production workflow.}
\resumeItemListEnd

\resumeSubHeadingListEnd

%=========================
% Open Source Projects
%=========================
\section{Open Source Projects}
\resumeSubHeadingListStart

\resumeProjectHeading
{Oombak}{November 2024 -- Present}
{\href{https://github.com/fuad1502/bilbob}{github.com/fuad1502/bilbob}}
\begin{justify}
	Oombak (/\textprimstress\textopeno mbak/, "waves" in Indonesian) is an
	interactive SystemVerilog simulator UI that runs on your terminal!
	Oombak is written in Rust \& \CC\ and uses
	\href{https://ratatui.rs/}{Ratatui},
	\href{https://www.veripool.org/verilator/}{Verilator}, and
	\href{https://www.sv-lang.com/}{slang} libraries, among others.
\end{justify}

\resumeProjectHeading
{Rubbler}{November 2023 -- March 2024}
{\href{https://github.com/fuad1502/rubbler}{github.com/fuad1502/rubbler}}
\begin{justify}
	Rubbler is a RISC-V assembler written in Rust. This library was written
	with the main purpose of embedding a simple RISC-V assembler inside of
	a RISC-V CPU test bench code written with Verilator.
\end{justify}

\resumeProjectHeading
{RVSV}{November 2023 -- March 2024}
{\href{https://github.com/fuad1502/rvsv}{github.com/fuad1502/rvsv}}
\begin{justify}
	RVSV is a SystemVerilog implementation of a 5-stage pipelined RISC-V
	CPU. Verification code is written in \CC\ using Verilator and Rubbler.
\end{justify}

\resumeProjectHeading
{Open Running Watch}{April 2024 -- May 2024}
{\href{https://github.com/fuad1502/open-running-watch-hw}{github.com/fuad1502/open-running-watch-hw}}
\begin{justify}
	An open source GPS running watch schematic \& PCB design.
\end{justify}
\resumeSubHeadingListEnd

%=========================
% Other Projects
%=========================
\section{Other Projects}
\resumeSubHeadingListStart

\resumeProjectHeading
{NAALG: Network Analyzer Algorithms \CC\ Library}{}
{Keysight Technologies}
\begin{justify}
	I initiated the development of NAALG due two things. First, I was
	tasked with developing a novel Time Domain Reflectometry (TDR)
	calculation from Vector Network Analyzer (VNA) measurements. Second, on
	a different project, I had just implemented a new calibration algorithm
	for an embedded VNA. Since both of this project involves calculation on
	VNA measurements, I realize that it would be nice to have a library
	that provides all of these VNA algorithms in one place. NAALG is
	similar to \href{https://scikit-rf.readthedocs.io}{scikit-rf}, but uses
	high performant implementation in \CC\ to enable usage in embedded
	real-time measurement applications.
\end{justify}

\resumeProjectHeading
{Custom Telecommunication Device (GMSK + TDM Transceiver)}{}
{Hariff Daya Tunggal Engineering}
\begin{justify}
	My team was tasked with developing a fully custom telecommunication
	device. My role was in developing the physical (PHY) and medium access
	control (MAC) layer on an \textit{All Programmable SoC (APSoC)}. The
	PHY layer is written in Verilog, while the MAC layer is written in C.
	Communication between the two layers uses Direct Memory Access (DMA).
\end{justify}

\resumeProjectHeading
{Automatic Identification System Search and Rescue Transponder}{}
{Labs247}
\begin{justify}
	Automatic Identification System (AIS) Search and Rescue Transponder
	(SART) is a radio device used to locate distressed vessels. My role was
	in developing all of the electronic and embedded software aspect of the
	device, and production. We use \href{https://freertos.org}{FreeRTOS}
	Real-Time Operating System to enable low power usage and multiple tasks
	management. Up till now the device is still in production.
\end{justify}

\resumeSubHeadingListEnd

%=========================
% Technical Skills
%=========================
\section{Technical Skills}
\begin{itemize}[leftmargin=0.15in, label={}]
	\small{\item{
	      \textbf{Programming Languages}{: Modern \CC, C, Rust, RISC-V Assembly, Go, JavaScript, Python, Bash, SQL, SystemVerilog, MATLAB} \\
	      \textbf{Developer Tools}{: CMake, Make, RPM, GDB, Git, Docker/Podman, GBS } \\
	      \textbf{Libraries / Frameworks}{: POSIX, GLib, OpenGL, React, Ratatui, FreeRTOS, Tizen NUI } \\
	      \textbf{Knowledge Domains}{: Compilers/Runtimes, Operating Systems, GNU/Linux, Embedded Systems, Digital/Analog/RF electronics, Schematic \& PCB Design }
	      }}
\end{itemize}

%=========================
% End
%=========================
\end{document}
